% for section-numbered lemmas etc., use "numberwithinsect"
\documentclass[a4paper,english,numberwithinsect]{eurocg19-submission}
\setcounter{errorcontextlines}{10000}
\usepackage{style/macros}

% the recommended bibstyle
\bibliographystyle{plainurl}

% -------------------------------------------------------------------
\usepackage{microtype} % if unwanted, comment out

%helpful if your graphic files are in another directory
%\graphicspath{{./graphics/}}

% Author macros::begin %%%%%%%%%%%%%%%%%%%%%%%%%%%%%%%%%%%%%%%%%%%%%%%%
\newcommand\polylog{{\rm polylog}}

% Author metadata::begin %%%%%%%%%%%%%%%%%%%%%%%%%%%%%%%%%%%%%%%%%%%%%%%%
\title{An Example Contribution for EuroCG 2019\footnote{Supported by
our friends.}}
%optional, in case that the title is too long;
%the running title should fit into the top page column
\titlerunning{A Contribution to EuroCG 2019}

\author[1]{Antonio Example}
\author[2]{Sylvia Veryknown}
\affil[1]{Important University\\
  \texttt{aexample@cs.important.de}}
\affil[2]{Department of Mathematics and Computer
        Science, University of Somewhere\\
  \texttt{sveryknown@math.somewhere.nl}}
%mandatory. First: Use abbreviated first/middle names.
%Second (only in severe cases): Use first author plus 'et. al.'
\authorrunning{A. Example and S. Veryknown}
\ArticleNo{100}

% Author macros::end %%%%%%%%%%%%%%%%%%%%%%%%%%%%%%%%%%%%%%%%%%%%%%%%%


\begin{document}

\maketitle

\begin{abstract}
  This example file was adapted from Wolfang Mulzer's example file for EuroCG
  2018, which in turn was adapted from Bettina Speckmann's example file for
  EuroCG 2005. The current style uses two style-files, \texttt{eurocg19.cls}
  and \texttt{eurocg19-submission.cls}. The former was adapted from the
  LIPIcs-style, with kind permission from Dagstuhl publishing. The latter file
  provides proper line numbering and is adapted from the style file developed
  for SoCG 2019 by Michael Hoffmann. Thanks again to all of these people.

  Here you should probably write a concise, informative, and exciting abstract
  for your paper.
\end{abstract}

\section{Introduction}

\subsection{Problem Statement and Solution}

\subsubsection{Problem Setup}

We consider only the two-dimensional setting.
We assume \dots

\subparagraph{Precise Problem Formulation.}

Describe your problem as clearly as possibly, instead of the usual \dots

\begin{conj}
Could it really be like this?
\end{conj}
\begin{obs}
Probably not \dots
\end{obs}

\subsection{Basic Definitions}
\begin{definition}
Some things are just not definable \dots
\end{definition}

\subsection{Related Results from the Literature}

We improve upon the following well-known algorithm of Grace of
Florence~\cite{g-atpwog-2006} in the following way: \dots


\section{The New Algorithm}

%\section{Correctness}

\section{Complexity Analysis}

\begin{theorem}
This is the most important theorem.
\end{theorem}
\begin{proof}
It even comes with a proof \dots
\end{proof}

There should be some more text explaining 
research results in some additional sections, 
but since this is only an example file \dots

An enumeration:
\begin{itemize}
\item a
\begin{itemize}
  \item 0
  \item 1
\end{itemize}
\item b
\end{itemize}

\begin{lemma}
  % Some mathematics, not entirely convincing.
  The following formula holds
  for all integers $n>0$\textup:
\begin{equation}
  \label{eq:sum-formula}
  \sum_{i=1}^{n} i = \frac{n(n+1)}{2}
\end{equation}
\end{lemma}
\begin{proof} (Not entirely convincing)
Let \(T(n) := \frac{n(n-1)}{2}\) denote the claimed formula.
\begin{align}
  T(n) - T(n-1) &= \frac{n(n+1)}{2} - \frac{(n-1)n}{2}
\nonumber
  \\
                &= \frac{n(n+1) - (n-1)n}{2}
\nonumber
  \\
                &= \frac{n^2+n - (n^2+n)}{2} = \frac {2n}{2} = n
\label{eq:diff}
\end{align}
The induction basis $T(0)=\frac{0\cdot 1}2 = 0$, together with~\eqref{eq:diff},
establishes~\eqref{eq:sum-formula}.
\end{proof}

\begin{lemma}
  And then we also found this lemma, which we state without proof.
  \qed
\end{lemma}

\section{Conclusion}

\subparagraph*{Acknowledgments.} We thank the organizers for
the tasty cookies.

\bibliography{bibliography/all}

\end{document}
